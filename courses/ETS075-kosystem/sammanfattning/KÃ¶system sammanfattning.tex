\documentclass{article}
\usepackage{mathtools} 
\usepackage[swedish]{babel}
\usepackage[T1]{fontenc}
\usepackage[utf8]{inputenc}
\title{Kösystem sammanfattning}
 	\begin{document}
 	\maketitle
 	\section*{Definitioner}
 	{\bf Kund eller Jobb:} Det som anländer till ett system. De som ska betjänas. 
 	{\bf Ankomstintensitet:} Betecknas $\lambda$ och är frekvensen som kunder kommer till systemet.\\
 	{\bf Ankomstintensitet för en kund:} Betecknas $\beta$ och är intensiteten som en kund kommer till systemet.\\
 	{\bf Betjäningsintensitet:} Betecknas $\mu$ och är frekvensen som kunder lämnar systemet.\\
 	{\bf Erbjuden trafik:} Betecknas $\rho=\lambda/\mu$ och är kvoten mellan frekvensen som kunder kommer till systemet och frekvensen som kunder lämnar systemet.\\
 	{\bf Avverkad trafik:} Är kvoten mellan frekvensen som kunder kommer till systemet som inte spärras och frekvensen som kunder lämnar systemet. Detta beräknas genom $\lambda_{eff}/\mu$ \\
 	{\bf Tidsspärr:} Betecknas $p_m$ och är sannolikheten att ett system är fullt, dvs sannolikheten att vara i sista tillståndet.\\
 	{\bf Anropsspärr:} Betecknas $r_m$ och är sannolikheten att man anropar ett system som är fullt. Detta beräknas genom $r_m={\lambda_mP_m}/\sum_{i=1}^{m}\lambda_iP_i$\\
 	{\bf Lambdaeffektiv:} Betecknas $\lambda_{eff}$ och är frekvensen som kunder kommer till ett system som inte är fullt. $\lambda_{eff}=\lambda(1 - p_m)=\lambda_0p_0+\lambda_1p_1+\dots+\lambda_{m-1}p_{m-1}$\\ 	
 	{\bf Medelbetjäningstid:} Betecknas $\overline{x}=1/\mu$ och är medeltiden att betjäna en kund. \\
 	{\bf Medelväntetid:} Betecknas W och är medeltiden som en kund få stå i kö, och är $W=T-\overline{x}$.\\
	{ \bf Medelantalet kunder:} som finns i ett system betecknas $E(N)$ eller $\overline{N}$ Detta kan beräknas med definitionen av väntevärde eller genom Littles sats. Med definition räknar man $E(N)=\sum_{i=1}^{k} kp_k$ där $p_k$ är sannolikheten för att vara i ett visst tillstånd i systemet. Ibland vill man också beräkna medelantalet kunder i betjänaren eller i kön. Då använder man att totala antalet kunder/jobb i systemet är detsamma som summan av de i kön och de hos betjänaren, dvs $\overline{N}=\overline{N_q}+\overline{N_s}$.\\
	{ \bf Medeltid i systemet:} som finns i ett system betecknas $T$. Medeltiden i systemet kan beräknas med följande samband:  $T=W+\overline{x} = W + 1/\mu = \overline{N}/\lambda_{eff}$
	\section*{Satser}
	{ \bf Littles sats:} Säger att medelantalet kunder i systemet är lika med produkten mellan ankomstintensiteten för de som ej spärras av ett system och medeltid i systemet. Dvs $\overline{N}=\lambda{eff}\cdot T$. Detta ger också följande formuleringar:\\
	$\overline{N_s}=\lambda{eff}\cdot \overline{x}$ och $\overline{N_q}=\lambda{eff}\cdot W$
	\section*{Olika sorters system}
	{\bf M/M/1-system:}\\
	{\bf M/M/m-system:}\\
	{\bf M/M/1-upptagetsystem:}\\
	{\bf M/M/m-upptagetsystem:}\\
	{\bf M/G/1-system:}\\
	{\bf M/G/m-system:}\\	
 	\section*{Rita tillståndsdiagram}
	Ett tillståndsdiagram ska innehålla tre saker: {\bf Ankomstintensitet, betjäningsintensitet och tillstånd.}\\\\
	{\bf Ankomstintensitet:} Betecknas $\lambda$ eller $\beta$ och är frekvensen som kunder kommer till systemet. $\lambda$ används när det är samma ankomstintensitet hela tiden för alla tillstånd. $\beta$ används när ankomstintensiteten beror på antalet kunder, dvs $\beta$ är ankomstintensiteten som varje kund kan bidra med.\\
	{\bf Betjäningsintensitet:} Betecknas $\mu$ och är frekvensen som kunder lämnar systemet.\\
	{\bf Tillstånd:} Det ska finnas tillstånd från noll kunder i systemet upp till antalet köplatser och betjänare. Dvs 1 köplats och 1 betjänare innebär tre tillstånd. Ett för 0 kunder i systemet, ett för 1 kund i betjänaren och ett för 1 kund 1 betjänaren och 1 i kön. Dessa tillstånd betecknas 0,1,2. Om man har mer än 1 betjänare måste man tänka på att betjäningsintensiteten ökar. Dvs ifall det finns 2 betjänare så kommer alla tillstånd som har två betjänare att ha dubbelt så hög betjäningsintensitet än i tillstånden där man endast använder en betjänare. \\
	\section*{Beräkna tillståndssannolikheter}
	Tillståndssannolikheter är sannolikheten att befinna sig i ett visst tillstånd i systemet. Det finns många sätt att räkna ut tillståndssannolikheterna men den mest välanvända i denna kurs är snittmetoden. Med snittmetoden kan man beräkna sannolikheten att vara i alla olika tillstånd i tillståndsdiagrammet. Man föreställer sig ett snitt mellan två tillstånd och skriver upp ekvationen som kommer av att alla pilar genom snittet. Man brukar då först försöka skriva om alla tillstånd så att alla sannolikheter är uttryckt i sannolikheten att vara i tillstånd 0. Sedan utnyttjar man att $\sum_{i=0}^{k}p_k = 1$ och löser ut $p_0$. Om vi byter ut $p_0$ i den vanliga uppgiften så har vi svaret.

\section*{Könät}
Ett könät består av flera kösystem(flera noder). Här kommer lite om några beräkningar man kan göra på dessa könät:
\begin{enumerate}
  \item Skriv upp allt som anges i texten. Gör alltid detta! Och skilj på olika sorters kösystem!
  \item Beräkna ankomstintensiteter för varje nod.Utnyttja att det som kommer in i en nod är det som kommer ut ur en nod. Kommer det in olika intensiteter till en nod så summera de som kommer in.
  \item Beräkna erbjuden trafik för varje nod. Utnyttja att $\rho=\lambda/\mu$ 
  \item Beräkna medelantalet kunder för varje system. Detta görs med hjälp av littles sats($\overline{N} = \lambda_{eff}\cdot T$) eller om systemet är ett M/M/1 system(med oändlig kö) kan formeln $\rho/1-\rho$ användas.
  \item Beräkna medeltiden i systemet för varje nod. Här kan också littles sats användas $T =\overline{N}/\lambda_{eff}$. Om systemet är ett upptagetsystem(dvs inga köplatser, W=0) så är medeltiden i en nod lika med medelbetjäningstiden($T = W + \overline{x} = 0 + 1/\mu$).
  \item Beräkna medelantalet kunder som går igenom systemet eller en viss nod i systemet. Det totala medelantalet i systemet är summan av medelantalet vid varje nod($\overline{N_{tot}} = \overline{N_1} + \ldots + \overline{N_k} $).
  \item Beräkna medeltiden som går igenom hela systemet eller genom en viss nod i systemet. För att ta reda på medeltiden i hela systemet så måste man beräkna medeltiden för alla möjliga vägar och multiplicera detta med sannolikheten att ta just rätt väg. Dvs T = E(T)=E(T|Väg 1)P(Väg 1)+E(T|Väg 2)P(Väg 2)+\ldots+E(T|Väg k)P(Väg k).
  \item Undersök vilken nod som riskerar att bli instabil först. Stabilitetskravet säger att $\rho=\lambda/\mu\leq1$ för att vara stabilt. Uttryck därför alla stabiliteter med ett och samma $\lambda$ och kolla vad detta $\lambda$ blir då $\rho=1$. Jämför sedan vilket $\lambda$ som blir minst. Detta betyder att den klarar max av ankomstintensiteten $\lambda$ innan det blir instabilt.
\end{enumerate}
 	\end{document}
